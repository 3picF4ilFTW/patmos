\section{Data Representation}

Data words in memories are stored using the big-endian data representation, this
also includes the instruction representation.

\section{Register Usage Conventions}

The register usage conventions for the general purpose registers are as follows:
\begin{itemize}
  \item \texttt{r0} is defined to be zero at all times. This is actually not just a convention,
  but implemented by the hardware.
  \item \texttt{r1} and \texttt{r2} are used to pass the return value on
        function calls. \\
        For 64 bit results, the high part is stored in \texttt{r1},
        the low part in \texttt{r2}.
	32 bit results are returned using \texttt{r1} only.
  \item \texttt{r3} through \texttt{r8} are used to pass arguments on function
        calls. \\
        For 64 bit arguments, the high part is stored first,
        followed by the low part.\\ E.g., for a 64 bit argument passed in
        \texttt{[r3,r4]}, the high part is in \texttt{r3}, the low part
        in \texttt{r4}.
  \item \texttt{r29} is used as temp register.
  \item \texttt{r30} is defined as the frame pointer and
        \texttt{r31} is defined as the stack pointer for the \emph{shadow}
        stack in global memory.
        The use of a frame pointer is optional, the register can freely be
        used otherwise.
	\texttt{r31} is guaranteed to always hold the current stack pointer and
        is not used otherwise by the compiler.
  \item \texttt{r1} through \texttt{r19} are caller-saved \emph{scratch}
        registers.
  \item \texttt{r20} through \texttt{r31} are callee-saved \emph{saved}
        registers.
\end{itemize}

The usage conventions of the predicate registers are as follows:
\begin{itemize}
  \item all predicate registers are callee-saved \emph{saved} registers.
\end{itemize}


\stefan{We would gain in most cases by making predicates caller-saved, since predicates are usually not
used across function calls, this saves up to 6 instructions per call.
Predicate live ranges over calls only happen with single-path and if-conversion. However, having only caller-saved predicates
makes if-conversion of calls too costly (and too complex). A nicer solution would be to have p1-p4 caller-saved and
p5-p7 callee saved. I got this to work (do not alias s0 with p1-p7 in RegInfo.td, mark s0 and p5-p7 as callee saved
in RegInfo.cpp, and in processBeforeCalleeSavedScan set s0 as used when any of p5-p7 is used), but this would require the
if-converter to use callee-saved predicates when converting calls, i.e., to change the predicate of the condition.
Sounds easy, but is actually far from trivial to hack into the if-converter, and only works if the if-converter runs before the
prologue-inserter, so we stay with callee saved registers for now and live with the costs of spilling s0 at nearly every call.}
\daniel{It would also break the single-path conversion pass.}

The usage conventions of the special registers are as follows:
\begin{itemize}
  \item \texttt{s0}, representing the predicate registers, is a callee-saved
        \emph{saved} register.
  \item The stack cache control registers \texttt{ss} and \texttt{st} are
        callee-saved \emph{saved} registers.
  \item The return information registers \texttt{s7-s10} (\texttt{srb},
        \texttt{sro}, \texttt{sxb}, \texttt{sxo}) are callee-saved
        \emph{saved} registers.
  \item All other special registers are caller-saved scratch registers and
        should not be used across function calls.
\end{itemize}



\section{Function Calls}
\label{sec:function_calls}

Function calls have to be executed using the \texttt{call}
instruction that automatically prefetches the target function to the method
cache and stores the return information in the special registers \texttt{srb}
and \texttt{sro}.

The register usage conventions of the previous section indicate which registers
are preserved across function calls.

The first $6$ arguments of integral data type are passed in registers, where
64-bit integer and floating point types occupy two registers. All other
arguments are passed on the \emph{shadow} stack via the global memory.

\fb{We could pass arguments via the stack cache, however, this would work only
for functions with a fixed number of arguments. The calling convention for
variadic function would then differ from the standard conventions. We should
probably also introduce a size limit on how many arguments should be passed via
the stack cache. Thus, for a first shot I decided to keep the conventions
simple.}
\martin{I would like to see arguments passed via the stack cache. No
reason to go via main memory (when the number is fixed.}

\martin{BTW: an on-chip stack, with single cycle access, is not so different from
a large register file. Maybe using a sliding window to keep the number of
addressing bits down. Shall we merge the stack cache with the register file?
Are we than redoing a stack (Java) processor?}

When the return function base \texttt{srb} and the return offset \texttt{sro}
needs to be saved to the stack, they have to be saved as the first elements of
the function's stack frame, i.e., right after the stack frame of the calling
function.
%
Note that in contrast to \texttt{br} and \texttt{brcf} the return offset refers
to the next instruction after the \emph{delay slot} of the corresponding
\texttt{call} and can be implementation dependent (cf.\ the description of the
\texttt{call} and \texttt{ret} instructions).

\section{Sub-Functions}
A function can be split into several sub-functions. The program is only allowed to use
\texttt{br} to jump within the same sub-function. To enter a different sub-function,
\texttt{brcf} must be used. It can only be used to jump to the first instruction of a
sub-function.

In contrast to \texttt{call}, \texttt{brcf} does not provide link information.
Executing \texttt{ret} in a sub-function will therefore return to the last \texttt{call}, not to the last \texttt{brcf}.
Function offsets however are relative to the \emph{sub-function} base, not to the surrounding function.
The function base register \texttt{r30} must therefore be set to the base address
of the current \emph{sub-function} for calls inside sub-functions.

A sub-function must be aligned and must be prefixed with a word containing the size of the sub-function,
like for a regular function. If a function is split into sub-functions, the first sub-function
must also be prefixed with the size of the first sub-function, not with the size of the whole function.

There are no calling conventions for jumps between sub-functions, for the compiler
this behaves just like a regular jump.
% except that the base register texttt{srb} must be updated if the
% sub-function contains calls.


% TODO: examples

\stefan{We need to define how the call and local branch with cache fill get the size of the code to load. A word
containing the size at the target address, or prior to the target address? (including or excluding the size word?)}

\jack{I'd store the code size at the target address and have the
method itself start at an offset +4 or +8 from that place. I also
suggest storing the stack size here too. Could fit into the same
32-bit word (since I'd guess the local memory size limits us to
$<64$k code and $<64$k stack in a single method anyway).}

\martin{Agree for plain C code where there is no data structure
further describing a function. In the JOP JVM I have the luxury
to have those sizes as part of the virtual method dispatch table.
One indirection is there needed anyway, so why not having those
data there.}

\section{Stack Layout}

All stack data in the global memory, either managed by the stack cache or using
a frame/stack pointer, grows from top-to-bottom. The use of a frame pointer is
optional.

Unwinding of the call stack is done on the stack-cache managed stack frame,
following the conventions declared in the previous subsection on function calls.

\section{Interrupts and Context Switching}

\daniel{TODO update}

Interrupt handlers may use the shadow stack pointer \texttt{r31} to spill registers
to the shadow stack. Interrupt handlers must ensure that all special registers
that might be in use when the interrupt occurs % MS: there is probably no 'might' WP: not all special registers are used, e.g., s11-s15
are saved and restored.
%
Here is an example of storing and restoring the context for context switching.
\begin{verbatim}
sub $r31 = $r31, 56
swc [$r31 + 0] = $r20  // free some registers
swc [$r31 + 1] = $r21
swc [$r31 + 2] = $r22
swc [$r31 + 3] = $r23
mfs $r20 = $s0
mfs $r21 = $sm
mfs $r22 = $sl  // by now any mul should be finished
mfs $r23 = $sh
swc [$r31 + 4] = $r20
swc [$r31 + 5] = $r21
swc [$r31 + 6] = $r22
swc [$r31 + 7] = $r23
mfs $r22 = $ss  // read out cache pointers, spill
mfs $r23 = $st
sub $r22 = $r23, $r22
sspill $r22   // spill the memory, s5 == s6 now
swc [$r31 + 8] = $r22  // store the stack pointer
swc [$r31 + 9] = $r23  // store stack size
mfs $r20 = $srb  // store return info
mfs $r21 = $sro
mfs $r22 = $sxb
mfs $r23 = $sxo
swc [$r31 + 10] = $r20
swc [$r31 + 11] = $r21
swc [$r31 + 12] = $r22
swc [$r31 + 13] = $r23
swc [$r31 + 14] = $r30 // store frame pointer
...

// restore
lwc $r20 = [$r31 + 4]
lwc $r21 = [$r31 + 5]
lwc $r22 = [$r31 + 6]
lwc $r23 = [$r31 + 7]
mts $s0 = $r20
mts $sm = $r21
mts $sl = $r22
mts $sh = $r23
lwc $r22 = [$r31 + 8] // restore the stack
lwc $r23 = [$r31 + 9]
mts $ss = $r23
mts $st = $r23   // set top = spill and fill from memory
sens $r22
lwc $r20 = [$r31 + 10] // restore return registers
lwc $r21 = [$r31 + 11]
lwc $r22 = [$r31 + 12]
lwc $r23 = [$r31 + 13]
mts $srb = $r20  // restore return infos and registers
mts $sro = $r21
mts $sxb = $r22
mts $sxo = $r23
lwc $r20 = [$r31 + 0]
lwc $r21 = [$r31 + 1]
lwc $r22 = [$r31 + 2]
lwc $r23 = [$r31 + 3]
lwc $r30 = [$r31 + 14]
xret
add $r31 = $r31, 52
nop
nop
\end{verbatim}

\todo{Check why add is 52 and not 56 as in the beginning.}

